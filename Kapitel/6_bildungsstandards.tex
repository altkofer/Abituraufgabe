%----------------------------------------eingefuegt Axel Enders 26.10.2023-----------------------------------------------------------------------------------------
\newpage
\chapter{Bildungsstandards und die KmK}\label{KmK}
Die Kultusministerkonferenz (KmK) hat beschlossen,\footnote{KMK $\textrm{--}$ St{\"a}ndige Konferenz der Kultusminister in der Bundesrepublik Deutschland (2004): Bildungsstandards im Fach Pysik f{\"u}r den Mittleren Schulabschluss. Beschluss vom 16.12.2004. Bonn: KMK.} Bildungsstandards zu erarbeiten mit dem Ziel, die Einheitlichkeit und Vergleichbarkeit von Zeugnissen und Abschl{\"u}ssen zwischen den Bundesl{\"a}ndern zu vereinbaren und dadurch ein H{\"o}chstma{\ss} an Mobilit{\"a}t von Fachkr{\"a}ften innerhalb der Bundesrepublik zu erreichen und  zur Gleichwertigkeit der Lebensverh{\"a}ltnisse in ganz Deutschland beizutragen. Die Bundesl{\"a}nder haben sich verpflichtet, diese Standards zu implementieren und anzuwenden. Diese (m{\"u}hsam errungenen) Bildungsstandards werden allgemein als nationale Kompetenzziele angesehen und sollen als Messlatte f{\"u}r den Erfolg des Schulunterrichts, u.a. im Fach Physik, dienen. 
\mip
Mit dem Erwerb des mittleren Schulabschlusses verf{\"u}gen die SuS {\"u}ber naturwissenschaftliche \emph{Kompetenzen} im Allgemeinen, und {\"u}ber physikalische Kompetenzen im Besonderen. 
\mip
\leftskip=0.5cm \rightskip=0.5cm {\emph{{\textbf{Definition von Kompetenzen}} nach Weinert:\footnote{Weinert, F. E., Vergleichende Leistungsmessung in Schulen $\textrm{--}$ eine umstrittene Selbstverst{\"a}ndlichkeit, in: Weinert, F. E. (Hrsg.), Leistungsmessungen in Schulen, 2001} die bei Individuen verf{\"u}gbaren oder durch sie erlernbaren kognitiven F{\"a}higkeiten und Fertigkeiten, um bestimmte Probleme zu 
l{\"o}sen, sowie die damit verbundenen motivationalen, volitionalen und sozialen Bereitschaften und F{\"a}higkeiten, um die Probleml{\"o}sungen in variablen Situationen erfolgreich und verantwortungsvoll nutzen zu k{\"o}nnen.}} 
\mip
%  \tabto{12em} \hangindent=4.62cm
\leftskip=0cm \rightskip=0cm Die in den folgenden vier Kompetenzbereichen festgelegten Standards beschreiben die notwendige physikalische Grundbildung. Zu jedem Kompetenzbereich sind drei Anforderungsbereiche  (= Merkmale von Aufgaben, die verschiedene Schwierigkeitsgrade innerhalb ein und derselben Kompetenz abbilden), \textbf{I}, \textbf{II}, \textbf{III}, angegeben. 
\mip
{\textbf{Fachwissen}} \tabto{7em} \hangindent=2.7cm Physikalisches Fachwissen, wie es durch die vier Basiskonzepte (Materie, Wechselwirkungen,System, Energie) charakterisiert wird, beinhaltet Wissen {\"u}ber Ph{\"a}nomene, Begriffe, Bilder, Modelle und deren G{\"u}ltigkeitsbereiche sowie {\"u}ber funktionale Zusammenh{\"a}nge und Strukturen. Als strukturierter Wissensbestand bildet das Fachwissen die Basis zur Bearbeitung physikalischer Probleme und Aufgaben. \\ \emph{\textbf{I} Wissen wiedergeben; \textbf{II} Wissen anwenden; \textbf{III} Wissen transferieren und verkn{\"u}pfen}
\mip
{\textbf{Erkenntnisgewinnung}} \tabto{12em} \hangindent=2.7cm Physikalische Erkenntnisgewinnung ist ein Prozess, der durch die T\"{a}tigkeiten \emph{Wahrnehmen}, \emph{Ordnen}, \emph{Erkl{\"a}ren bzw. Pr{\"u}fen} sowie \emph{Modelle bilden} beschrieben werden kann. \\ \emph{\textbf{I} Fachmethoden beschreiben; \textbf{II} Fachmethoden nutzen; \textbf{III} Fachmethoden problembezogen ausw{\"a}hlen und anwenden}
\mip
{\textbf{Kommunikation}} \tabto{9em} \hangindent=2.7cm Informationen sach- und fachbezogen erschlie{\ss}en und austauschen. Dazu geh{\"o}ren das angemessene Verstehen von Fachtexten, Graphiken und Tabellen, der Umgang mit Informationsmedien sowie das Dokumentieren von gewonnenem Wissens. Zur Kommunikation sind die Sprech- und Schreibf{\"a}higkeit in der Alltags- und der Fachsprache, das Beherrschen der Diskussion und Techniken der Pr{\"a}sentation erforderlich. \\ \emph{\textbf{I} Mit vorgegebenen Darstellungsformen arbeiten; \textbf{II} Geeignete Darstellungsformen nutzen; \textbf{III} Darstellungsformen selbst{\"a}ndig ausw{\"a}hlen und nutzen}
\mip
{\textbf{Bewertung}} \tabto{7em} \hangindent=2.7cm Physikalische Sachverhalte in verschiedenen Kontexten erkennen und bewerten. Dazu geh{\"o}ren das Heranziehen physikalischer Denkmethoden und Erkenntnisse zur Erl{\"a}uterung, zum Verst{\"a}ndnis und zur Bewertung physikalisch-technischer und gesellschaftlicher Entscheidungen, sowie die Kenntnis der Grenzen naturwissenschaftlicher Sichtweisen. \\ \emph{\textbf{I}  Vorgegebene Bewertungen nachvollziehen; \textbf{II} Vorgegebene Bewertungen beurteilen und kommentieren; \textbf{III} Eigene Bewertungen vornehmen}



