\chapter{Abituraufgabe 2024-1}\label{Denk}

\begin{aufgabe}
	\textbf{1. Aufgabenname}
	\begin{enumerate}
		\item Unteraufgabe \hfill K3, S2 / I
  \section*{Regulus}

Der Stern Regulus im Sternbild Löwe ist 77,6 Lj von der Erde entfernt. Er besitzt den 3,24-fachen Radius der Sonne und die relative Leuchtkraft $L^* = \frac{L}{L_\odot} = 138$.

\begin{enumerate}
    \item[a)] Zeigen Sie, dass die Oberflächentemperatur von Regulus $11 \times 10^3$ K beträgt. Nehmen Sie dazu an, dass seine Oberfläche ein Schwarzer Strahler ist.

    \item[b)] Ergänzen Sie im nebenstehenden Hertzsprung-Russell-Diagramm (Abb. 1) die fehlenden Spektralklassen und tragen Sie die Positionen von Sonne und Regulus sowie den Bereich der Roten Riesen und den Bereich der Weißen Zwerge ein. Nennen Sie zwei Merkmale der Sterne, die sich im gleichen Entwicklungsstand wie Regulus befinden.

    \item[c)] Die absolute Helligkeit von Regulus beträgt $-0,57$. Zeigen Sie, dass sich rechnerisch für die scheinbare Helligkeit 1,31 ergibt. Durch Messungen ergibt sich für die scheinbare Helligkeit jedoch 1,36. Nennen Sie einen möglichen Grund für diese Abweichung.

    \item[d)] Abb. 2 zeigt die spektrale Intensitätsverteilung der Sonne. Ergänzen Sie den für das Auge sichtbaren Bereich und tragen Sie qualitativ den Verlauf der zu Regulus gehörigen Kurve ein. Geben Sie an, ob Regulus dem Betrachter rötlich oder bläulich erscheint.
\end{enumerate}
		\item Unteraufgabe \hfill K3, S2 / I
	\end{enumerate}
	Zwischentextiii
	\begin{enumerate}[resume]
		\item Unteraufgabe \hfill K3, S2 / I
  \section*{Die Sonne}

Am 12. August 2018 startete die Raumsonde „Parker Solar Probe“ der NASA zur Sonne. Das Raumfahrzeug wird der Sonne näher kommen als alle seine Vorgänger. Man verspricht sich von dieser Mission insbesondere neue Erkenntnisse über die Verhältnisse in der äußeren Atmosphäre (Korona) unseres Zentralgestirns.

\begin{enumerate}
    \item[a)] In einer vereinfachten Modellvorstellung lässt sich der Aufbau der Sonne in Zonen gliedern. Fertigen Sie eine beschriftete Skizze der Zonen an und beschreiben Sie die wesentlichen Vorgänge, die den Energietransport vom Zentralbereich der Sonne bis zur Oberfläche ermöglichen.

    \item[b)] Durch Fusion von vier Protonen zu einem Heliumkern wird in der Sonne Energie freigesetzt. Geben Sie die Reaktionsgleichung an; die Angabe von Zwischenprodukten ist dabei nicht nötig. Zeigen Sie, dass bei diesem Prozess eine Energie von 24,7 MeV frei wird, und berechnen Sie die Energie $E$, welche die Fusion von 1,0 kg Wasserstoff liefert. \[zur Kontrolle: $E = 5,9 \times 10^{14}$ J\]

    \item[c)] Eine der ersten Theorien ging von der Annahme aus, dass die Energie der Sonne aus der Verbrennung fossiler Brennstoffe stammt. Bei vollständiger Verbrennung von 1,0 kg fossilen Brennstoffs werden $5,0 \times 10^7$ J Energie freigesetzt. Untersuchen Sie, ob eine solche Verbrennung die Energie für den gesamten bisherigen Lebenszeitraum der Sonne von 4,5 Milliarden Jahren hätte liefern können, wenn deren Leuchtkraft für diesen Zeitraum als konstant vorausgesetzt wird.

    \item[d)] Schätzen Sie ab, um wie viel Prozent die Masse des Wasserstoffs in 4,5 Milliarden Jahren durch Kernfusion gesunken ist. Gehen Sie davon aus, dass die Sonne zur Zeit ihrer Entstehung zu 75 \% aus Wasserstoff bestand.

    \item[e)] Neben der Umstellung von fossilen auf regenerative Energieträger könnte die Kernfusion in Zukunft eine weitere Option zur Energieversorgung der Welt darstellen. Bewerten Sie diese Option unter Berücksichtigung der bisherigen Teilaufgaben sowie eines weiteren, nichtphysikalischen Aspekts.

    \item[f)] Die Sonde „Parker Solar Probe“ soll am 24. Dezember 2024 ihren sonnennächsten Punkt erreichen. Ihre Entfernung zur Sonne beträgt dann $8,86 \ R_\odot$. Danach bewegt sich die Sonde für einige Zeit auf einer elliptischen Bahn mit einer Umlaufdauer von 88 Tagen um die Sonne. Ein ebener Hitzeschild, der immer zur Sonne hin gerichtet ist, schützt die Sonde vor zu starker Hitze durch die Sonnenstrahlung.

    \begin{enumerate}
        \item Berechnen Sie die maximale Entfernung $r_{\text{max}}$ der Sonde vom Sonnenmittelpunkt in Sonnenradien. \[zur Kontrolle: $r_{\text{max}} = 1,6 \times 10^2 \ R_\odot$\]

        \item Schätzen Sie unter Annahme einer Albedo von 0,5 die Temperatur des Hitzeschilds im sonnennächsten Punkt nach oben ab. Nehmen Sie hierzu stark vereinfachend an, dass der Hitzeschild die aufgenommene Energie ausschließlich in die sondenabgewandte Richtung wieder abgibt. Nennen Sie die weiteren Annahmen, die Sie dabei machen. \[zur Kontrolle: $T = 1,63 \times 10^3$ K\]

        \item Der Hitzeschild hielt in Tests Temperaturen von bis zu 1650 °C stand. Beurteilen Sie, inwieweit die Temperaturbeständigkeit des Hitzeschilds sowie die Bahnform geeignet gewählt sind.

        \item In der Nähe des Perihels muss die Sonde die Sonnenkorona durchqueren, deren Temperatur etwa eine Million Grad Celsius beträgt. Erklären Sie, dass die Sonde die Korona trotz der hohen Temperatur unbeschadet durchqueren kann.

\end{enumerate}
  
	\end{enumerate}
\end{aufgabe}
\maketitle





\section{Fachliche Grundlagen}

\section{Musterlösung}

\begin{hinweis}
	Hinweise werden so gesetzt.
\end{hinweis}
