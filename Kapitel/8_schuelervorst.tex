\chapter{Sch{\"u}lervorstellungen und Konzeptwechsel}\label{Schuelervorstellungen}


\section{Konstruktivismus}\label{Konstruktivismus}

Der Konstruktivismus ist eine lerntheoretische Perspektive, die das Lernen nicht als eine Folge des Lehrens beschreibt, sondern als eigenst"{a}ndige Konstruktionsleistung des Lernenden. Lernen wird als individueller, aktiver Prozess verstanden, bei dem neue Informationen auf Basis von Vorerfahrungen und bereits vorhandenen Wissensstrukturen interpretiert und integriert werden. Somit betont der konstruktivistische Ansatz  die aktive Rolle des Lernenden im Lernprozess. Insbesondere die Extremform des Konstruktivismus, der \emph{radikale} Konstruktivismus, bewirkt  seit den 1990-er Jahren immer wieder erregte Diskussionen, die gut in Referenz\footnote{Werner Jank und Hilbert Meyer, Didaktische Modelle, Cornelsen, 14. Auflage (2021)} dargestellt sind. F\"{u}r Lehrpersonen ist es wichtig zu erkennen, dass Lernende neues Wissen aus ihrer sehr eigenen, individuellen Perspektive betrachten und aufnehmen und somit Lernende, bei gleichem Wissensangebot, zu unterschiedlichem Lernergebnis und auch zu unterschiedlichen Verst\"{a}ndnisschwierigkeiten kommen k\"{o}nnen.
\bip
Im konstruktivistischen Ansatz stehen folgende Prinzipien im Vordergrund:

\begin{itemize}
\item \textbf{Aktive Wissenskonstruktion:} Lernende bauen neues Wissen aktiv auf, indem sie es mit ihrem bestehenden Wissen verkn\"{u}pfen und selbst Bedeutungen entwickeln.
\item \textbf{Selbstgesteuertes Lernen:} Lernende sind selbst f\"{u}r ihren Lernprozess verantwortlich, was Autonomie und Eigenverantwortung f\"{o}rdert.
\item \textbf{Soziale Interaktion:} Wissen wird oft in sozialen Kontexten konstruiert, durch Diskussionen, Zusammenarbeit und Austausch mit anderen.
\item \textbf{Kontextgebundenes Lernen:} Wissen wird als kontextgebunden betrachtet, d.h., dass Lerninhalte in realen oder realit\"{a}tsnahen Situationen vermittelt werden, um das Verst\"{a}ndnis und die Anwendbarkeit zu f\"{o}rdern.
\item \textbf{Fehler als Lernchance:} Fehler werden als wichtiger Teil des Lernprozesses gesehen, da sie dazu beitragen, Denkprozesse zu reflektieren und zu korrigieren.
\end{itemize}

\bip\bip
\section{Sch{\"u}lervorstellungen $\textrm{--}$  Begriff und Ursachen}
Vorunterrichtliche Erfahrungen und Vorstellungen beeinflussen das Denken der SuS. Die SuS besitzen in der Regel bereits vor Einsetzen des Fachunterrichts in Physik diverse individuelle Vorerfahrungen und teilweise breit gestreutes, unvollst{\"a}ndiges Vorwissen {\"u}ber naturwissenschaftliche Sachverhalte oder $\textrm{--}$ allgemeiner $\textrm{--}$ nat{\"u}rliche Ph{\"a}nomene. Sch{\"u}lervorstellungen sind die Hauptursache von Lernschwierigkeiten, denn der Lernende verarbeitet Unterrichtsinhalte auf Grundlage pysikalisch unangemessener Denkweise. In einem Unterricht, in dem zulasten begrifflicher Erkl{\"a}rungen viel gerechnet wird, treten falsche Sch{\"u}lervorstellunge oft nicht zutage. 

Es gibt ein Netz von verwandten Begriffen, die unterschiedliche Intentionen beinhalten:

\begin{itemize}
\item Von einem ganzen \emph{Fehlkonzept} kann man sprechen, wenn eine Fehlvorstellung nicht nur einer
schnell-intuitiven Einsch\"{a}tzung, sondern einem ,,stabilen, weil erfolgreichen'' pers\"{o}nlichen
Theoriesystem entspringt, das eine gr\"{o}{\ss}ere Gruppe von Ph\"{a}nomenen vermeintlich richtig deutet.
\item
Im Begriff \emph{Pr\"{a}konzept} ist der fr\"{u}hzeitige Erwerb in einem au{\ss}erphysikalischen Kontext betont.
\item
Der Begriff \emph{Alltagskonzept} unterstreicht, dass sich Fehlvorstllungen aufgrund von Beobachtungen
oder Informationen aus dem Alltagsleben zustandekommen.
\item
Es gibt Tendenzen, die Vorsilbe ,,Fehl-'' zu vermeiden, da sie zu stark eine Art Vorab-Aburteilung beinhaltet.
\end{itemize}

\bip


Zu den typischen Ursachen von Sch{\"u}lervorstellungen geh{\"o}ren:

\begin{itemize}
\item
\emph{Umgangssprache}. Beispiele: {\glqq}Die W{\"a}rme breitet sich nach Norddeutschland aus{\grqq} $\textrm{--}$  W{\"a}rme ist als beweglicher Stoff angesehen. {\glqq}Der Stromverbrauch in Deutschland steigt.{\grqq} $\textrm{--}$  anstatt Nutzung elektrischer Energie. {\glqq}Seine Kraft reichte nicht mehr aus, um den Schlusssprint zu gewinnen.{\grqq} $\textrm{--}$ Kraft als Eigenschaft oder gespeicherter Vorrat einer Person. 
\item
\emph{Wahrnehmungsmuster}. Beispiele: Darstellung von Atomen, Atomkernen,Elektronen als Kugeln in den Medien. Die Feldlinien sind das Feld, zwischen den Feldlinien ist nichts. Das gleiche gilt f{\"u}r Strahleng{\"a}nge aller Art. Das Modell wird  als Realit{\"a}t angenommen.
\item
\emph{Stark wirkende Erfahrungen}. Beispiele: erfahrene Kr{\"a}fte bei der Kurvenfahrt mit dem Auto. Es gibt also eine Kraft, die mich nach au{\ss}en tr{\"a}gt... . "Bewegung braucht Kraft" (Aristoteles)
\item
\emph{Weitere Ursachen:} Zu wenig Beobachtungs-Erfahrungen, zu schnell gefertigte Eigendeutung, Begriffsunklarheiten, fehlende Einsicht in Modellcharakter (Idealisierung) einer physikalischen Theorie. 
\end{itemize}

Zu den am weitest verbreiteten Informationsquellen z{\"a}hlen Eltern, Gro{\ss}eltern, Geschwister, Freunde, Erzieher (z.B. Kindergarten), Massenmedien (z.B. Die Sendung mit der Maus), Filme (auch Computerfilme), Comics und Anime,  Solche inkompletten Kenntnisse und Vorerfahrungen werden in der Mehrheit unreflektiert auf neue Sachverhalte angewendet.
\mip
Solch u.U. kognitiv und affektiv fest verankerten Vorstellungen legen Sch{\"u}ler mit Betreten des Physikfachraumes beileibe nicht ab. Diese Vorstellungen werden in der fachdidaktischen Literatur mit Namen wie \emph{Alltagsvorstellungen}, \emph{Sch{\"u}lervorstellungen} oder auch \emph{Pr{\"a}konzepte} belegt. Der Physiklehrer muss bei seiner Unterrichtsplanung  vorhandene Pr{\"a}konzepte antizipieren und ihnen durch seinen Unterricht entgegenwirken. Werden falsche Pr{\"a}konzepte nicht ausger{\"a}umt bzw. beseitigt, entwickeln sich daraus \emph{Misskonzepte} oder sog. \emph{Fehlvorstellungen}.

Fehlvorstellungen erweisen sich meist als sehr stabil, da sie beispielsweise
\begin{itemize}
\item fr\"{u}hzeitig,
\item durch unmittelbare Erfahrung eines Ph\"{a}nomens,
\item durch \"{U}bernahme aus der direkten sozialen Umgebung,
\item verbunden mit einem einfachen Erkl\"{a}rungsmodell
\item positive Affekte bei einem AHA-Erlebnis
\end{itemize}
erlernt werden.

\bip\bip
\section{Tests zum Erfassen von Misskonzepten}

\begin{itemize}
\item Force Concept Inventory (FCI) dient dem Erfassen des grundlegenden Verst\"{a}ndnisses der Konzepte in der Newtonschen Mechanik
\item The Force and Motion Concept Evaluation (FCME) ist dem FCI sehr \"{a}hnlich, baut aber Vorstellungen zur Energie und Energieerhaltung mit ein
\item Der Mechanics Baseline Test (MBT) bietet 26 Aufgaben mit je 5 Auswahlantworten, stellt formal h\"{o}here Anforderungen als FCI und FMCE. Stellt auch einige Aufgaben, deren L\"{o}sung rechnerische Absch\"{a}tzungen erfordern und sollte auch nicht vor dem Mechanik-Unterricht eingesetzt werden.
\item Testfragen zur Erfassung von Misskonzeptein in der klassischen Mechanik nach Nachtigall (Nachtigall, Dieter, 1987, 8. Skizze in: Skizzen zur Physikdidaktik, Frankfurt/M., S. 144ff.)
\end{itemize}

\bip\bip
\section{Strategien zum Erreichen eines Konzeptwechsels}
Es geht nicht darum, physikalisch falsche Konzepte durch richtige zu ersetzen. Sondern es geht darum eine neue, \emph{physikalische Sichtweise} zu vermitteln und deren Unterschiede zum \emph{Alltagsdenken} herauszustellen. Folgende grundlegenden Strategien stehen zum Erreichen eines Konzeptwechsels zur Verf{\"u}gung:

\begin{itemize}
\item \textbf{Dialog.} Thematisierung der Fehlvorstellungen. SuS sollen sich ihrer eigenen Ideen und Vorstellungen im Gespr{\"a}ch bewu{\ss}t werden.  Ein grunds{\"a}tzliches Problem beim Thematisieren von Sch\"{u}lervorstellungen besteht darin, dass diese dadurch aufgewertet werden und in Erinnerung bleiben. 


\item \textbf{Schlagartiges Umdenken, Bruch, Konfliktstrategie, bzw. Akkomodation.} Hierbei geht es um eine Konfliktstrategie f{\"u}r einen diskontinuierlichen Lernweg. Als Grundlage wird ein stabiles Netz an unpassenden Konzepten gesehen, da{\ss} man grunds{\"a}tzlich ersch{\"u}ttern muss. Man versucht hier, kognitive Konflikte zu erzeugen, indem man mit Aspekten beginnt, die mit den Sch{"u}lervorstellungen nicht vereinbar sind. So soll eine Unzufriedenheit mit vorhandenem Wissen erzeugt werden, sowie der Wunsch nach einem korrekten Konzept.  Ein solcher kognitiver Konflikt k\"{o}nnte beispielsweise durch ein Experiment erzeugt werden, bei dem die Sch\"{u}ler falsche Vorhersagen \"{u}ber den Ausgang treffen. Die Strategie zielt auf ein schnelles Umdenken bzw. einen Paradigmenwechsel ab. 

\item \textbf{Allm\"{a}hliches Entwickeln, Umdeuten, Assimilation.}  Ein kontinuierlicher, bruchloser Wandlungsprozess wird hingegen erreicht, wenn man versucht, an bestehende, ausbauf\"{a}hige Sch\"{u}lervorstellungen anzukn\"{u}pfen, um von da aus Schritt f\"{u}r Schritt zu den gew\"{u}nschten Vorstellungen hinzuf\"{u}hren. Hierbei werden Wahrnehmungen und Beobachtungen zu einem vorhandenen Wahrnehmungsschema zugeordnet, welches sich dadurch nur leicht ver\"{a}ndert. Dies wird erreicht durch Ankn\"{u}pfen, Abgrenzen, Umgehen, schrittweises Ausbauen oder durch Umdeuten von bestehenden, ausbauf\"{a}higen  Sch\"{u}lervorstellungen. Hierzu m\"{u}ssen gen\"{u}gend Ankn\"{u}pfungspunkte an entwicklungsf\"{a}hige vorhandene Sch\"{u}lervorstellungen vorhanden sein. 

\item \textbf{{\"U}berbr{\"u}ckungsstrategie} Ist wohl eher als eine Mischform aus diskontinuierlicher und kontinuierlicher Strategie zu sehen. Es geht hierbei darum, eine Gedankenbr\"{u}cke zu schaffen, um eine falsche Vorstellung umzudeuten. Beispielsweise besteht eine g\"{a}ngige Sch\"{u}lervorstellung in der Mechanik beim Verst\"{a}ndnis des 3. Newtonschen Gesetzes. Ein auf einem Tisch ruhendes Buch \"{u}bt demnach eine Kraft auf den Tisch aus, der Tisch jedoch \"{u}bt keine Kraft auf das Buch aus. Die Gedankenbr\"{u}cke best\"{u}nde darin, die Tischplatte durch eine d\"{u}nne biegsame Membran zu ersetzen oder durch eine auf Kompressionsfedern gelagerte d\"{u}nne Platte. Man erkennt dann recht einfach, dass diese mechanischen Federn nat\"{u}rlich auch eine Kraft auf das Buch aus\"{u}ben m\"{u}ssen. Generell sind Beispiele f\"{u}r \"{U}berbr{\"u}ckungsstrategien rar.
\end{itemize}

Folgende Bedingungen m{\"u}ssen nach Poser und Strike erf{\"u}llt sein, damit es {\"u}berhaupt zum Konzeptwechsel kommt:

\begin{itemize}
\item die SuS m{\"u}ssen mit ihrem vorhandenen Konzept unzufrieden sein
\item das neue Konzept muss wenigstens bis zu einem gewissen Grad verstanden sein
\item das neue Konzept muss intuitiv einleuchtend sein
\item das neue Konzept muss hilfreich sein in neuen Situationen und Ph{\"a}nomenen
\end{itemize}

\bip\bip
\section{Fazit}
Die meisten in der Literatur vorgeschlagenen Unterrichtsstrategien gehen ungef{\"a}hr so:

\begin{enumerate}
\item 
SuS machen eigene Erfahrungen mit den Ph{\"a}nomenen, z.B. durch eigenes Experimentieren oder eignes Vorhersagen eines Versuchsausgangs
\item  
Aktivierung der Sch{\"u}lervorstellungen durch Herstellen eines Konfliktes bzw. Umgehung der Sch{\"u}lervorstellungen in einer Aufbaustrategie
\item
Lehrkraft bringt die wissenschaftliche Sicht ein, die SuS nicht selber entdecken k{\"o}nnen. Nutzen wird im Unterricht diskutiert.
\item
Festigung der neuen Sichtweise durch Anwendung auf weiterer Beispiele
\item
kritischer R{\"u}ckblick auf den Lernprozess. Vergleich von Sch{\"u}lervorstellungen mit physikalischen Vorstellungen
\end{enumerate}

Die Lehrperson sollte sich immer bewusst sein, dass sich Sch\"{u}lervorstellungen hartn\"{a}ckig halten. Auch wenn diese im Unterricht explizit thematisiert und remediert wurden, wird allzu h\"{a}ufig festgestellt, dass zu sp\"{a}terer Zeit, in anderem Zusammenhang oder auch in gleicher Situation, die alte, falsche Vorstellung wieder dominiert. Es braucht Zeit und \"{U}bung, um eine physikalische Denkweise durchzusetzen.

\bip\bip
\section{Beispiele}
Die im folgenden getroffenen Aussagen beschreiben
Fehlvorstellungen, sie sind also {\bf falsch}.

\pph{Allgemein}
\begin{itemize}
\item Physik ist schwer, langweilig, alltagsfern, trocken.
\item
Es werden direkte Proportionalit\"{a}ten angenommen, obwohl ein
nichtlinearer Zusammenhang besteht.
\item
Begriffe im Umfeld von Energie: Verbrauch, Erzeugung,
Produktion, Lieferung, Verschwendung.
\end{itemize}

\pph{Mechanik}
\begin{itemize}
\item Tr\"{a}gheitssatz:
Die Bewegung eines K\"{o}rpers wird dadurch ausgel\"{o}st oder
aufrechterhalten, dass eine Kraft wirkt.
Historisch entspricht dies dem \"{U}bergang Aristoteles --- Galilei.
\item Der Bremsweg ist direkt proportional zur Anfangsgeschwindigkeit.
\item Begriff der Fallgeschwindigkeit eines K\"{o}rpers oder eines
Stoffes.
(Beispiel: F\"{u}hrung durch die N\"{u}rnberger Burg/Tiefer
Brunnen, Wasser hat eine Fallgeschwindgkeit von \SI{10}{\meter\per\second}).
\item Zentripetalkraft: Wenn ein K\"{o}rper kreisf\"{o}rmig bewegt wird,
so ist eine nach au{\ss}en gerichtete Kraft daf\"{u}r notwendig.
\item Zentrifugalkraft: Wird ein gleichf\"{o}rmig-kreisf\"{o}rmig bewegter K\"{o}rper losgelassen, so fliegt er radial davon.
\item
Beim Festhalten eines (schweren) Gegenstandes (B\"{u}chertasche) wird Arbeit an ihm verrichtet.
\item
Bei einer Schraubenfeder sind angreifende Kraft und L\"{a}nge der Schraubenfeder
direkt proportional zueinander.
\item R\"{u}cksto{\ss}prinzip: Eine Rakete fliegt deshalb, weil sie sich an dem umgebenden Medium (beispielsweise Luft) abst\"{o}{\ss}t.
\item
Oberfl\"{a}chenspannung: Das Wasser bildet eine Haut auf seiner Oberfl\"{a}che.
\item
Gleichsetzung von Volumen und Masse: \SI{1}{\liter} entspricht \SI{1}{\kilogram} (insbesondere bei Fl\"{u}ssigkeiten).
\item
Gleichsetzung von Masse und Gewicht(-skraft): \SI{100}{\gram} ,,entspricht'' \SI{1}{\newton}.
\end{itemize}

\pph{Elektromagnetismus}
\begin{itemize}
\item
In Bezug auf die el.\ Leitf\"{a}higkeit unterscheidet man lediglich ,,Leiter'' und ,,Nichtleiter (Isolatoren)''.

\item Begriffe, die im Alltag und in der Technik verwendet werden, im genauen
fachlichen Wortsinne aber falsch sind:
\begin{quote}
Stromverbrauch(er), Stromsparen,
Stromerzeuger,
Stromlieferung, Stromz\"{a}hler,\dots
\end{quote}

\item
Nur in einem geschlossenen Stromkreis kann Stromfluss zustande kommen.
(\"{U}berbetonung dieses Aspekts z.B.\ in der Grundschule, Problem bei Auf- oder Entladung, Elektronik, Erde).

\item
Der Transport elektrischer Energie ist an die Bewegung der
Ladungstr\"{a}ger gebunden (vgl.\ \cite{DuitEnergie}).
\item
Die f\"{u}r die Gef\"{a}hrlichkeit von Elektrizit\"{a}t entscheidende Gr\"{o}{\ss}e
ist die el.\ Spannung (Es ist vielmehr die el.\ Stromst\"{a}rke).
\item
Schaltet man zu einer Gl\"{u}hbirne eine weitere parallel, so
brennt diese dunkler. (Diese Beobachtung kann man bei
Verwendung einer Trockenbatterie tats\"{a}chlich machen; die
Ursache ist der Innenwiderstand der Batterie).
\item
Die Bewegung von Elektronen erfolgt mit Lichtgeschwindigkeit.
\item
Bei Gewitter: ,,Buchen sollst Du suchen''.
\item
Alle Metalle sind magnetisch.
\item
In der N\"{a}he des Nordpols (der Pole) befindet sich ein gro{\ss}er Eisenberg.
\end{itemize}

\pph{W\"{a}rmelehre}

\begin{itemize}
\item
W\"{a}rme ist ein den warmen K\"{o}rper durchdringender Stoff
(Historisch: Dalton, Querschnitt Physik und Technik, S.\ 261)
\item
Temperatur ist eine additive Gr\"{o}{\ss}e: Sch\"{u}ttet man (gleiche)
Mengen Wasser der Temperaturen \SI{10}{\celsius} und \SI{20}{\celsius}
zusammen, so erreicht man eine Temperatur von \SI{30}{\celsius}.
\item
W\"{a}rme riecht.
\item
Wasserdampf ist sichtbar, Nebel oder beim Kochen aufsteigende Schleier bestehen aus Wasserdampf.
\item
Die in den Jahreszeiten unterschiedliche Erw\"{a}rmung kommt von den unterschiedlich langen
Lichtwegen durch die Erdatmosph\"{a}re (Blacky Fuchsberger in ,,Das fliegende Klassenzimmer'').
\end{itemize}

\pph{Optik}
\begin{itemize}
\item
Die optische Wahrnehmung erfolgt durch eine Aktivit\"{a}t des
Sehenden. Er muss einen Blick auf den Gegenstand werfen.
\item
Die mit den Augen wahrgenommene Welt ist in eine Art Lichtmeer, -bad oder -stoff getaucht.
\item
Bei Dunkelheit ist die Welt mit einer Art Dunkelstoff oder -nebel ausgef\"{u}llt, weshalb Lichtstrahlen nicht mehr
durchdringen k\"{o}nnen.
\item
Einen Lichtstrahl kann man sehen (Vgl.\ Wolkenloch oder Waldlichtung).
\item
Farbe ist eine unab\"{a}nderliche Eigenschaft von Stoffen oder K\"{o}rpern.
\item
Die Farbe ,,Schwarz''.
\item
Das Himmelsblau hat seine Ursache in der Reflexion des Lichts in den Weltmeeren.
\item
Ein Stab erscheint beim Eintauchen nach unten geknickt (Zeichnung des gebrochenen Lichtstrahls).
\item
Stahleng\"{a}nge: Der Strahlengang bei einer Abbildung durch Sammellinsen wird als Lochkamera-Strahlengang gezeichnet.

\item
Das Spiegelbild ist reell, es ist auf der spiegelnden Fl\"{a}che lokalisiert.

\item
Man kann einen Regenbogen aus der N\"{a}he genauer betrachten.

\item Der Regenbogen hat die Form eines (kreisf\"{o}rmigen Teils eines) Rings.
\end{itemize}

\pph{Astronomie}
\begin{itemize}
\item
Historisch: Die Erde ist eine Scheibe, alle Gestirne bewegen
sich \"{u}ber diese Scheibe weg. Gegenst\"{a}nde (oder Menschen) m\"{u}ssten
auf der anderen Seite herunterfallen.
\item
Die bei den verschiedenen Mondphasen auftretende Abdunkelung
eines Teils der Mondes ist durch den Erdschatten zu erkl\"{a}ren.
\item
Die Jahreszeiten und ihre unterschiedlichen Temperaturen kommen
durch die wechselnde Entfernung zur Sonne der Erde beim
Durchlaufen ihrer elliptischen Bahn um die Sonne zustande.
(Vgl.\ auch das Beispiel oben: Blacky  Fuchsberger)

\item
Der Andromedanebel ist ein Nebel oder Materiedunst im Weltraum.
\item
Die Erde dreht sich nach Westen.
\end{itemize}

\pph{Atomphysik}
\begin{itemize}
\item
Das Bohr'sche Atommodell: Die Elektronen laufen auf Bahnen um den Kern und haben stets einen festen Ort
und eine feste Geschwindigkeit.
\item
Elektronen sind (als bl\"{a}uliche Strahlen) sichtbar.
\item
Heisenberg'sche Unbestimmtheitsrelation.
\end{itemize}

\pph{Chemie}
\begin{itemize}
\item
Wachs brennt (Richtig: Wachsdampf brennt)
\end{itemize}

\pph{Gr\"{o}{\ss}enordnungen}
\begin{itemize}
\item
Geschwindigkeit des Elektronendrifts bei Stromfluss in Metallen.
\item
Ein Liter passt auf keinen Fall in einen W\"{u}rfel mit Seitenl\"{a}nge
\SI{1}{\deci\meter}.
\item
Die durchschnittliche Geschwindigkeit eines Weltklassesprinters
ist wesentlich gr\"{o}{\ss}er als \SI{36}{\kilo\meter\per\hour}.
\end{itemize}

\pph{Technik}
\begin{itemize}
\item
Verwechslung von Kenn- und Betriebsdaten (z.B.\ bei einer Gl\"{u}hbirne).
\end{itemize}



