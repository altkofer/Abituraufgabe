\chapter{Zum Begriff des Lernziels}\label{Lernziel}\label{Ziele}

\begin{quote}
	Im Physik- und Chemieunterricht sollen die Sch\"{u}ler jene Wissens-, Denk-, und Handlungsschemata erlernen, die sie f\"{u}r eine selbstgesteuerte, flexible Bew\"{a}ltigung und Gestaltung ihrer gegenw\"{a}rtigen und zuk\"{u}nftigen Lebenssituation brauchen k\"{o}nnen.
\end{quote}

Jedes planm\"{a}{\ss}ige Handeln erfordert eine Zielsetzung:
So hat sich in der lerntheoretisch orientierten Unterrichtsforschung im Laufe der Zeit der Gedanke
der Zielformulierung durchgesetzt.
\mip
Der Begriff {\bf Lernziel} deutet dabei auf ein Endverhalten
der Sch\"{u}ler hin. Es geht zun\"{a}chst weniger um die Art oder die Methodik
des Lernprozesses.
\mip
Ein Weg kann klarer ausgew\"{a}hlt und leichter beschritten werden,
wenn das Ziel bekannt ist.

\begin{itemize}
	\item Lernziele bilden f\"{u}r den Lehrenden den entscheidenden Rahmen f\"{u}r die
	Unterrichtsplanung und -umsetzung.
	\item Lernziele geben den Lernenden eine Orientierung über die von ihnen regelmäßig erwarteten Handlungsmuster und Lernerfolge.
	\item In Lernzielen werden die Erwartungen einer Gesellschaft
	(einschlie{\ss}lich Wirtschaft, Verb\"{a}nde, Kirchen, Hochschulen) an
	die Institution ,,Schule'' formuliert.
	\item Traditionen, neue Entwicklungen, Weltanschauungen oder
	Ideologien schlagen sich in den Lernzielen nieder.
	Beispiele: NewMaths, Umweltbewegung, Europa.
	\item
	Sie spiegeln deshalb den st\"{a}ndigen gesellschaftlichen
	Wechselprozess aus Bewahrung und Ver\"{a}nderung wieder.
	\item
	Im Begriff ,,Lernziel'' begegnen sich zwei grundlegende
	Dimensionen von Unterricht:
	
	\begin{itemize}
		\item
		Auftrag zur Bildung der Pers\"{o}nlichkeit
		(Anthropologische oder personale Dimension)
		\item
		Vermittlung fach(wissenschaft)licher Inhalte (Sachliche
		oder inhaltliche Dimension)
	\end{itemize}

\end{itemize}

Lernziele erfahren vielerlei Klassifizierungen, die in einer
Diskussion \"{u}ber Lernziele bzw.\ in einer auf Lernzielen
gr\"{u}ndenden Unterrichtsarbeit hilfreich sind.

%\bip\bip
%\section{Wahl der grammatischen Form}
%\begin{itemize}
%\item
%{\bf Aussages\"{a}tze} \q Der Sch\"{u}ler kennt das Ohm'sche Gesetz.
%\item
%{\bf Soll-S\"{a}tze} \q Der Sch\"{u}ler soll mit den Grundgr\"{o}{\ss}en der
%Elektrizit\"{a}tslehre vertraut sein.
%\item
%{\bf Substantive} \q F\"{a}higkeit, die
%Begriffe ,,Kraft'' und ,,Arbeit'' zu unterscheiden.
%\end{itemize}
%
%Ausf\"{u}hrliche S\"{a}tze bergen die Gefahr der st\"{a}ndigen Wiederholung
%und Inflation der gleichen Formulierungen (,,soll kennen'').
%
%Kurzs\"{a}tze reduzieren menschliches Verhalten auf technisch anmutende
%Begriffsformeln.

\bip\bip
\section{Funktionen von Lernzielen}
\begin{itemize}
	\item
	Lernziele unterst\"{u}tzen das Bestreben, Inhalte deutlicher
	im Hinblick auf Rahmenbedingungen (Lernen, Unterricht,
	gesellschaftlich, Umwelt) im Unterricht umzusetzen.
	\mip
	Beispiel: Der Inhalt ,,Energieerhaltungssatz'' erf\"{a}hrt 
	eine Ausgestaltung durch das Lernziel:
	\begin{quote}
		Die Sch\"{u}ler sollen Einsicht in die Bedeutung des
		Energieerhaltungssatzes f\"{u}r technische, wirtschaftliche und
		\"{o}kologische Prozesse gewinnen.
	\end{quote}
	\item
	Lernziele erm\"{o}glichen eine kritisch-systematische Reflexion
	\"{u}ber den Unterricht: Die Gretchenfrage lautet
	\begin{quote}
		Haben die Schüler die gesteckten Lernziele erreicht?
	\end{quote}
	und nicht
	\begin{quote}
		Wurde der Stoff ausreichend behandelt?
	\end{quote}
	
	\item
	Lernziele bilden eine solide Basis f\"{u}r die Evaluation von Unterricht.
	\item
	Lernziele stellen in den Mittelpunkt den Lernprozess der
	Sch\"{u}ler und nicht die Fachinhalte (Lerninhalte) oder
	die Unterrichtsmethodik (Lehrziele).
	\item
	Lernziele stellen ein Instrument f\"{u}r die Diskussion \"{u}ber
	schulische Erziehung und Unterricht bereit und bilden
	daher eine M\"{o}glichkeit zur Verst\"{a}ndigung von Lehrern,
	Sch\"{u}lern, Eltern, Didaktikern, Bildungspolitikern
	(Beispiel: Weltanschaulicher Unterricht, Sexualkunde).
	\item
	Der normative Charakter von Lernzielen erm\"{o}glicht es, den
	gesellschaftlichen Konsens \"{u}ber die Schule in den Unterricht zu
	transportieren.
	
	\mip
	Lehrpl\"{a}ne (in Bayern) werden vom Kultusministerium verordnet,
	und im Amtsblatt ver\"{o}ffentlicht.
	In der Zeitschrift ,,schule \& wir'' (jeweils Heft
	3/September) erscheint eine Liste der jeweils g\"{u}ltigen
	Lehrplan-Fassungen f\"{u}r alle F\"{a}cher und Schularten.
	Sie sind auch im Internet (Zugang \"{u}ber {\tt www.schule.bayern.de})
	abrufbar.
\end{itemize}

\section{Lernzieltaxonomien}
Der Grad, zu dem ein Lernender einen Lernstoff durchdrungen hat, lässt sich durch eine sog. \emph{Taxonomie} klassifizieren. 

\bip\bip
\section{Differenzierung bzgl.\ des Allgemeinheitsgrads}

Die zeitliche Tragweite korrespondiert im allgemeinen mit dem
Allgemeinheitsgrad.

Hier liegt das Modell des (fr\"{u}heren) curricularen Lehrplans
(CuLP) in Bayern zugrunde.
Die Begriffe sind aber auch heute noch anzutreffen.

\mip
Die zeitliche Tragweite f\"{u}r die einzelnen Ebenen ist nicht genau
anzugeben. So gibt es unterschiedliche Auffassungen dar\"{u}ber,
ob Feinziele innerhalb einer Unterrichtsstunde oder innerhalb
einer Unterrichtsphase
anzustreben sind.

\begin{enumerate}
\item
Leitziele: Umfassen des obersten Bereichs der p\"{a}dagogischen
Aufgaben und Absichten.
Sie leiten sich direkt aus dem Bildungs- und Erziehungsauftrag
der Verfassung an die Schule ab.
(Vgl.\ Bayerische Verfassung Art. 131, Bayerisches
Erziehungs- und Unterrichtsgesetz Art 1/2)

\mip
Beispiele: Studierf\"{a}higkeit, Berufsf\"{a}higkeit, Allgemeinbildung,
Bew\"{a}ltigung der
Lebenswelt, gesellschaftliche Verantwortung.

\item
Richtziele sind genauere teilweise fachspezifische
Ausformulierungen der Leitziele.

\mip
Beispiele: \\
Einblick in die Arbeits- und Denkweisen der Physik \\
Fertigkeiten im Umgang mit physikalischen Verfahren in Handwerk
und Technik. \\
\"{U}berblick \"{u}ber physikalische Ph\"{a}nomene und die ihnen
innewohnenden Gesetzm\"{a}{\ss}igkeiten.

\item
Grobziele beschreiben eindeutig, aber nicht im Detail,
die angestrebten Lernergebnisse innerhalb eines Faches.

\mip
Beispiele: \\
Verst\"{a}ndnis f\"{u}r den Begriff der ,,Arbeit''. \\
Kenntnis der grundlegenden Begriffe der Bewegungslehre. \\
Fertigkeit im Umgang mit dem L\"{o}tkolben. \\
Einblick in die Funktionsweise eines Generators. \\
Freude beim Experimentieren mit elektronischen Bauteilen.

\item
Feinziele: (Unterrichtsziele, Teilziele)
Sie differenzieren den Unterricht in kleinste Einzelziele.

\mip
Beispiele \\
F\"{a}higkeit den Widerstand eines Bauelements zu berechnen. \\
Kenntnis der Knotenregel. \\
Fertigkeit, die Durchschnittsgeschwindigkeit bei einer Radtour
zu bestimmen.
\end{enumerate}

\bip\bip
\section{Differenzierung bzgl.\ des Grades der Operationalisierung}

Hier liegt das Zielebenenmodell (ZEM) nach Eigenmann und
Strittmacher (1972) zugrunde.

\begin{enumerate}
\item Leitideen: Bezugsrahmen f\"{u}r Lernziele.
\item Dispositionsziele:
Erkennen, Bereitsein, F\"{a}higsein, Durchschauen, Verstehen
\item
Operationalisierte (Verwerklichte) Lernziele:
Auf Endverhalten abhebende Lernziele werden dahingehend
modifiziert, dass das erw\"{u}nschte Verhalten des
Sch\"{u}lers nach au{\ss}en beobachtbar (erfa{\ss}bar, me{\ss}bar) wird.
OLZs bed\"{u}rfen einer ausf\"{u}hrlichen Beschreibung und sollen keine
Interpretationsspielraum mehr zulassen.
Bedingungen an OLZe (nach Mager und Gagne):
\begin{enumerate}
\item
Benennung des Endverhaltens, das direkt beobachtbar sein muss
(Psychologisch steckt dahinter: Der Behaviorismus)
\item
Eindeutige Bezeichnung des Gegenstandes, auf den sich
Lernziel bezieht
\item
Beschreibung der Rahmenbedingungen, Voraussetzungen,
Hilfsmittel (z.B.\ Formelsammlung, Taschenrechner).
\item
Angabe eines Beurteilungsma{\ss}stabes  f\"{u}r das als ausreichend
geltende Verhalten.
\end{enumerate}  %%%%%%%%%%%%%%%%%%%

So sind beispielsweise die Lernzielverben aus der
Lernzielmatrix ersetzt durch
\begin{quote}
schreiben, auswendig aufsagen, identifizieren, unterscheiden,
vergleichen, Aufgabe l\"{o}sen, benennen,\dots
\end{quote}

\mip
Beispiel: \\
Der Sch\"{u}ler soll Aufbau und Wirkungsprinzipien eines
Elektromotors verstehen.

\mip
Operationalisiert:
Der Sch\"{u}ler soll das Modell eines Elektromotors aufbauen k\"{o}nnen,
seine wichtigsten Teile benennen und ihre Funktion erkl\"{a}ren k\"{o}nnen.

\mip
Noch weiter (hat schon die Form einer Lernzielkontrolle)
\begin{quote}
Der Sch\"{u}ler soll das Leybold-Modell eines Elektromotors
innerhalb von 10 Minuten aufbauen und verschalten k\"{o}nnen, er
soll weiter die Begriffe ,,Stator, Rotor, Kommutator'' im
Modell zuordnen k\"{o}nnen und die Funktion jedes dieser Bauteile
in zwei S\"{a}tzen beschreiben k\"{o}nnen, (wobei zwei Fehler erlaubt
sind).
\end{quote}

Der Sinn wird schon fast konterkariert.
Nachteile: Atomisiertes Wissen und Faktenwissen wird stark betont.
Bildung wird auf die Erlernung einer Sammlung von
Verhaltensweisen reduziert.
\end{enumerate}

\bip\bip
\section{Klassifizierungen bzgl.\ verschiedener
                Dimensionen des menschlichen Verhaltens}
Lernziele lassen sich nach verschiedenen Dimensionen des
Gesamtspektrums menschlicher Verhaltensweisen einordnen.
Die darauf aufbauenden Klassifizierungen hei{\ss}en Taxonomien
(Einordnung, vgl.\ z.B.\ Biologie)

Sehr bekannt ist die Bloom'sche Taxonomie:
\begin{enumerate}
\item Kognitive Lernziele:
Denken, Wahrnehmung, Ged\"{a}chtnisbereich, intellektuelle
F\"{a}higkeiten
\mip
Weitere Unterteilung:
\begin{enumerate}
\item
Konzeptziele ($\to$ Ged\"{a}chtnis):
Erwerb von reproduzierbarem Wissen
\item
Prozessziele ($\to$ Denken und Urteilen): Methoden, Strategien,
Arbeitsweisen, Umstrukturierung, Probleml\"{o}sen.
\end{enumerate}

\item Affektive Lernziele:
Interessen, Einstellungen, Gef\"{u}hle, Werturteile
\mip
Beispiele:
Bewunderung des Farbenspiels eines Regenbogens, \\
Freude \"{u}ber das Funktionieren eines selbstgebauten
Elektromotors, \\
Sich ein Urteil \"{u}ber die Notwendigkeit und Gef\"{a}hrlichkeit der
Castor-Transporte bilden k\"{o}nnen.

\item Psychomotorische Lernziele:
Motorische (K\"{o}rperbewegungs-)Fertigkeiten
\mip
Beispiele:
Fingerfertigkeit beim Einf\"{a}deln eines Fadens, \\
Zeichnen einer Versuchsanordnung, \\
L\"{o}sen einer Schraube, \\
Justieren einer elektrischen Klingel, \\
Beherrschung des L\"{o}tens, \\
(M) Umgang mit Zirkel und Lineal.
\end{enumerate}

Physik: Alle drei Bereichen werden (in --- je nach
Schulart --- unterschiedlicher Gewichtung)
betroffen: In Gymnasien liegt der Schwerpunkt im
kognitiven Bereich, in der Mittelschule ist der psychomotorische
Bereich st\"{a}rker betont.

\mip
Eine absolute Trennung in diese drei Kategorien ist unnat\"{u}rlich
und unsinnig.
Die einzelnen Kategorien \"{u}berlagern sich und bedingen einander.
\mip
Kritisch anzumerken w\"{a}re hier, dass soziale Lernziele wie
Kooperationsf\"{a}higkeit, Kommunikationsf\"{a}higkeit fehlen.

\bip\bip
\section{Die Lernzielmatrix}

